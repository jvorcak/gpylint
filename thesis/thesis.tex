\documentclass[11pt,oneside,final]{fithesis2}

\usepackage[utf8]{inputenc}
\usepackage[slovak]{babel}
\usepackage[T1]{fontenc} 
\usepackage[plainpages=false,pdfpagelabels,unicode]{hyperref}
\usepackage{tabularx}
\usepackage{pifont}
\usepackage{graphicx}
\usepackage{float}
\usepackage{listings}
\usepackage{varioref}
\usepackage{multirow}
\usepackage{listing}
\usepackage[protrusion=true, expansion, kerning]{microtype}
\lstset{basicstyle=\ttfamily, breaklines=true, breakatwhitespace=false}

\thesistitle{Nástroje na analýzu kódu v jazyku Python a vizualizácia ich výstupu}
\thesissubtitle{Bakalárska práca}  
\thesisstudent{Ján Vorčák}
\thesiswoman{false}
\thesisfaculty{fi}  
\thesisyear{2012}
\thesisadvisor{Mgr. Marek Grác}
\thesislang{sk}
\emergencystretch4dd
\begin{document}
 \widowpenalty10000
 \clubpenalty10000
 \hfuzz2mm
 \FrontMatter  
 \ThesisTitlePage

\begin{ThesisDeclaration}
\DeclarationText
\AdvisorName
\end{ThesisDeclaration}

\begin{ThesisThanks}
\end{ThesisThanks}

\begin{ThesisAbstract}
\end{ThesisAbstract}

\begin{ThesisKeyWords}
\end{ThesisKeyWords}



\MainMatter  
\tableofcontents

\chapter{Úvod}

	Pri vývoji aplikácií musíme klásť dôraz nielen na samotné programovanie, ale najmä na analýzu a návrh systému, dôkladné otestovanie ale aj spätnú analýzu samotného kódu. Z tohto dôvodu vzniká pre takmer každý programovací či značkovací jazyk množstvo nástrojov, ktoré nám pomáhajú automaticky objaviť potencionálne chyby či porušenie konvencií v zdrojových kódoch. Často krát sú tieto nástroje síce efektívne a ľahko konfigurovateľné, no najmä výstupom nie príliš užívateľsky prívetivé. Asi najpoužívanejším nástrojom na analýzu kódu v jazyku Python je konzolová utilita Pylint. Pokiaľ si chceme vďaka tomuto programu vytvoriť obraz o väčšom projekte nestačí nám iba textový výstup, no musíme tieto dáta automaticky spracovať.
	
	Cieľom tejto bakalárskej práce je analyzovať nástroje na analýzu a vizualizáciu Python kódu ako aj vlastnosti jazyka ako je napríklad introspekcia, ktoré nám túto analýzu umožnujú. Z dôvodu vizualizácie je nutné taktiež naštudovať základy modelovania hierarchických štruktúr v UML. Výstupom bakalárskej práce je nástroj, ktorý poskytuje funkcionalitu, ktorú v existujúcich nástrojoch nenájdeme. Nástroj teda zjednoduší orientáciu najmä vo väčšom projekte pomocou vizualizácie dát najmä za pomoci vygenerovania interaktívnych UML diagramov. Program bude taktiež dopĺňať funkcionalitu, ktorú nám klasický program Pylint neumožnuje ako je napríklad filtrácia falošných poplachov. 

	Práca pozostáva z piatich kapitol. V druhej kapitole si predstavíme jazyk Python a jeho vlastnosti, neskôr rozanalyzujeme existujúce nástroje na analýzu kódu v jazyku Python a ich nedostatky. Pred implementačnou časťou si predstavíme knižnice, ktoré nám neskôr pomôžu pri implementácií nástroja ako je napríklad knižnica Gaphas na vykresľovanie grafiky v GTK programoch alebo samotný Pylint. Nasledovať bude kapitola o návrhu nástroja a samotná implementácia. Nakoniec zhodnotíme, či implementovaný nástroj spĺňa určené požiadavky či už po stránke funkcionality alebo škálovateľnosti a navrhneme zmeny na jeho zlepšenie.
 


%
%Nástroje pro analýzu kódu v Pythonu a vizualizace jejich výstupu.
%
%- Nastudujte možnosti introspekce v jazyce Python
%- Seznamte se s existujícími nástroji pro analýzu kódu v tomto programovacím jazyce
%- Seznamte se se základy modelování hierarchických struktur v UML
%- Vyberte si jeden z projektů pro analýzu kódu a vytvořte aplikaci, která bude schopná ukázat informaci o pozici nalezené chyby v rámci hierarchie modulů, tříd, funkcí a ve zdrojovém kódu.
%- Volitelně můžete implementovat například filtraci falešných poplachů
%

\chapter{O jazyku Python}

	\section{Čo je to Python?}
	Python je vysoko úrovňový, objektovo orientovaný dynamický jazyk, ktorý vytvoril holandský programátor Guido van Rossum ako následníka jazyka ABC.
Vyznačuje sa prehľadnou syntaxou, jednoduchosťou a modulárnosťou. Python podporuje viacero programátorských paradigmat, najmä objektovo orientované, imperatívne a čiastočne aj funkcionálne paradigma.

Existuje viacero implementácií jazka Python. Mezdi najznámejšie patria:
\begin{list}{•}{}
\item CPython
\item Jython
\item Python pre .NET
\item IronPython
\item PyPy
\end{list}

Najpožívanejšou a najpodporovanejšou je ale CPython. Každá z implementácií sa môže líšiť špecifickými informáciami mimo štandardnej Python dokumentácie.
Zdrojové kódy sú preložené do byte kódu a zvyčajne uložené v .pyc alebo .pyo súboroch, ktoré sú neskôr spúštané virtuálnou

	V štandardnej knižnici je dostupné množstvo datových typov ako napríklad reálne a komplexné čísla, celé čísla s neobmedzenou dĺžkou, znakové reťazce, zoznamy a slovníky. Datové typy su silno a dynamicky typované. Použitie nekompatibilného typu spôsobí vyvolanie výnimky. Python podporuje objektovo orientované programovanie vrátane viacnásobnej dedičnosti. Kód je sústredený do modulov a balíkov s možnosťou importovať špecifický modul, triedu, funkciu alebo iný objekt. Za účelom ošetrenia chýb Python podporuje vyvolávanie a odchytávanie výnimiek. Automatická správa pamäti nahrádza nutnosť manuálne alokovať a uvoľnovať pamäť v kóde.


	\section{Introspekcia v jazyku Python}
		Introspekcia je schopnosť preskúmať daný objekt a rozhodnúť o jeho identite, vlastnostiach a schopnostiach. Jazyk Python podporuje rozsiahlu introspekciu objektov. Medzi hlavné informácie, ktoré potrebujeme o objektoch v jazyku Python zisťiť patrí ich meno, typ, identita, vlastnosti, schopnosti a ich pôvod. Jazyk Python ponúka množstvo nástrojov, ktoré nám tieto vlastnosti umožnujú zisťiť. Môžeme ich rozdeliť do dvoch hlavných skupín. V prvej skupine sú funkcie či už zo štandardnej knižnice alebo z pomocných modulov akým je napríklad modul inspect, do druhej skupiny radíme atribúty objektov, ktoré priamo v objekte uchovávajú užitocné informácie.
	
		\paragraph{Funkcia dir()}
			Funkcia dir() je jedným z hlavných nástrojov introspekcie v jazyku Python a vracia zotriedený zoznam mien atribútov objektu, ktorý bol uvedený ako jej argument. Funkcia je súčasťou štandardnej knižnice, takže nemusíme importovať žiaden modul pre jej použitie. Na výpis funkcií zo štandardnej knižnice môžeme teda využiť samotnú funkciu dir().
			

\begin{lstlisting}[language=python]
In [1]: print dir(__builtin__)[-10:]
['str', 'sum', 'super', 'tuple', 'type', 'unichr', 'unicode', 'vars', 'xrange', 'zip']
\end{lstlisting}

			 V prípade, že je funkcia dir() použitá bez argumentov vracia zoznam mien, ktoré sú momentálne definované.

\begin{lstlisting}[language=python]
In [2]: print dir()
['In', 'Out', '_', '__', '___', '__builtin__', '__builtins__', '__name__', '_dh', '_i', '_i1', '_i2', '_ih', '_ii', '_iii', '_oh', '_sh', 'exit', 'get_ipython', 'help', 'quit']

\end{lstlisting}
	

		\paragraph{Atribút \_\_doc\_\_ }
			Atribút \_\_doc\_\_ obsahuje komentáre, ktoré popisujú objekt. V prípade, že prvý v module, triede alebo v metóde je znakový reťazec, je automatický považovaný za \_\_doc\_\_ atribút. V opačnom prípade sa jeho hodnota nastaví na None. Hodnoty \_\_doc\_\_ sa z dôvodu kompaktnosti nevkladajú do byte kódu.
			
\begin{lstlisting}[language=python]
In [3]: print __doc__.__doc__
str(object) -> string

Return a nice string representation of the object.
If the argument is a string, the return value is the same object.
\end{lstlisting}

		\paragraph{Atribút \_\_name\_\_ }
			Atribút \_\_name\_\_ obsahuje názov objektu odvodený z jeho typu. Niektoré objekty, ako napríklad objekty typu string tento atribút neobsahujú. Tento atribút obsahujú napríklad moduly. V prípade, že spúštame script priamo pomocou Python interpretu je atribút \_\_name\_\_ nastavený na '\_\_main\_\_' nakoľko Python interpreter je považovaný za hlavný modul. Tak isto v prípade, že Python skript spúštame z príkazového riadku. Je teda často používaný na rozpoznanie či daný modul len importujeme alebo priamo spúštame.
	
\begin{lstlisting}[language=python]		
In [4]: def my_function():
   ....:     pass
   ....: 

In [5]: my_function.__name__
Out[5]: 'my_function'

In [6]: __name__
Out[6]: '__main__'
\end{lstlisting}

		\paragraph{Funkcia type()}
		
		Funkcia type() zo štandardnej knižnice vracia typ jej argumentu. Ten vracia v podobe typového objektu, ktorý môže byť porovnávaný s typmi definovanými v module types.

\begin{lstlisting}[language=python]	
In [7]: type(my_function)
Out[7]: function

In [8]: type(1)
Out[8]: int
\end{lstlisting}


		\paragraph{Funkcia id()}
		
		Funkcia id() vracia unikátnu identitu objektu. Táto funkcia je užitočná nakoľko viacero premenných môže odkazovať na rovnaký objekt. Funkcia id() konkrétne vracia pamäťovú adresu objektu.

\begin{lstlisting}[language=python]	
In [9]: id('string')
Out[9]: 3078023712L

In [10]: id(id)
Out[10]: 3078187660L
\end{lstlisting}

		\paragraph{Funkcie hasattr() a getattr()}
		V prípade, že je potrebné zistiť prítomnosť alebo hodnotu atribútu štandardná knižnica ponúka funkcie hasattr() a getattr().

\begin{lstlisting}[language=python]	
In [11]: hasattr(id, '__name__')
Out[11]: True

In [12]: getattr(id, '__name__')
Out[12]: 'id'
\end{lstlisting}		

		\paragraph{Funkcia callable()}
		V niektorých prípadoch môžu objekty slúžiť na vyvolanie určitého druhu udalosti. Pomocou funkcie callable() sa dá overiť, či je daný objekt spustiteľný.

\begin{lstlisting}[language=python]	
In [13]: callable(id)
Out[13]: True

In [14]: callable(1)
Out[14]: False
\end{lstlisting}			


		\paragraph{Funkcie isinstance() a issubclass()}
		Funkcia isinstance() vracia hodnotu v závislosti na tom, či je objekt inštanciou danej triedy. Funkcia vracia True aj v prípade, že je objekt inštanciou jej predka.
		
		Funkcia issubclass() vracia True v prípade, že objekt reprezentujúci triedu je podtriedou druhého argumentu.

\begin{lstlisting}[language=python]	
class Person(object):
    pass
class Student(Person):
	pass
p=Person()
s=Student()

In [15]: isinstance(s, Student)
Out[15]: True

In [16]: isinstance(s, Person)
Out[16]: True

In [17]: issubclass(Student, Person)
Out[17]: True
\end{lstlisting}		
			
\chapter{Nástroje na analýzu kódu pre jazyk Python}
	\section{Nástroje analýzu kódu projektu}	
	\section{Nástroje na vizualizáciu projektu}
    
\chapter{Analýza nástroja Gaphas}

	\section{Základná charakteristika nástroja Gaphas}

		Gaphas predstavuje zoskupenie knižníc a nástrojov na vykresľovanie grafických objektov na určené elementy grafického rozhrania GTK. Je naprogramovaný v jazyku Python a vydaný pod ??? licenciou.

	\section{Popis Gaphas API}

Gaphas API využíva MVC návrhový vzor a môzme ho teda rozdeliť na 3 hlavné časti.
\begin{list}{•}{}
\item Model - canvas, items
\item View - view
\item Controller - tools
\end{list}

\subsection{Model}

    
\subsection{View}
View obsahuje všetko súvisiace so zobrazovaním a vykresľovaním jednotlivých elementov.

\subsection{Controller}

\begin{list}{•}{}
\item api/view
\item api/painters
\item api/gtkview
\end{list}
    
\section{Zhrnutie}    
    
\chapter{Projekt gpylint}
	\section{Aplikácia jednotlivých nástrojov}
	\section{Vyhodnotenie}	

\chapter{Záver}
\end{document}
